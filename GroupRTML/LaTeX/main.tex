\documentclass[12pt,a4paper]{article}
\usepackage{apacite}
\usepackage{graphicx}
\usepackage{amsmath}
\usepackage{geometry}
\usepackage{setspace}
\usepackage{enumitem}
\usepackage{hyperref}
\usepackage{lipsum} % For placeholder text, remove it for actual content

% Page setup
\geometry{margin=1in}
\setstretch{1.5}

% Title Page
\title{Critical Literature Review on [Challenges of using blockchain in higher education]}
\author{Group Members: \\
\textbf{Rusyaidi Azri Bin Mohd Zamri (1211108129)} \\
\textbf{Toh Jing Wei (1211109475)} \\
\textbf{Muhammad Qayyim Hannan Bin Mohamed Nazir (1211111107)} \\
\textbf{Lai Zi Xuan (1211109451)} \\
\\
Course: Research Methodologies for Computer Science \\
Assignment 1 - Critical Literature Review \\
Date of Submission: [September 10, 2024]}
\date{\today}

\begin{document}

\maketitle
\newpage

% Abstract
\begin{abstract}
    \noindent
    %[Write a brief summary of the research topic, objectives of the literature review, the scope of the review, key findings, and a brief conclusion. Keep it within 150-250 words.] 
    Blockchain is a new technology that revolutionize society in fields like finance, healthcare, and education. Although blockchain technology benefits education a lot, it is still resist by some Higher Education Institutions (HEIs). To solve this problem, this research tries to analyse the challenges and advantages of blockchain technology. Hence, this research mainly focus on challenges faced by blockchain technology in Higher Education Institutions (HEIs) and also advantages of using blockchain technology in Higher Education Institutions (HEIs). After the research, result shows that challenges faced by blockchain technology can be category into three dimensions, which are technological, organisational, and environmental. The benefits of blockchain technology includes improving efficiency, security, and credibility of education system, make education more transparent, and make education more accessible and trustworthy for students. While blockchain is still a new technology to education, researches had proved that it has the potential to make educational systems better and help Higher Education Institutes (HEIs) to reach their sustainability goals.
\end{abstract}
\newpage

% Table of Contents
\tableofcontents
\newpage

% 1. Introduction
\section{Introduction}
\begin{itemize}
    \item \textbf{Background:} Blockchain is a distributed, digital ledger that uses a series of “blocks” to hold data such as transaction dates, timings, amounts or participants.It said to be a secure way to protect sensitive data from being easily altered by anyone.Some study has said this helpful technology can revolutionize the way we keep records in the education system.
    \item \textbf{Research Problem/Question:} What are the major challengers need to be overcome for the use of blockchain in higher education?
    \item \textbf{Objectives of the Review:}
    
     - How will blockchain technology be implement in education system
     
     - Find out what are the trends in the challenges for blockchain system

     - What are researches thoughts on blockchain use in education
    \item \textbf{Scope of the Review:} This report will mainly cover the research on blockchain in education section and will explain what are the challenges that will encounter.
\end{itemize}


% 2. Literature Review
\section{Literature Review}

\subsection{Overview of Selected Papers}
\begin{itemize}
    \item \textbf{Paper 1:} Blockchain System in the Higher Education 
    \begin{itemize}
        \item \textbf{Citation and Authors:} Ricardo Raimundo, Alberico Rosario, 2021 \cite{p1}
        \item \textbf{Research Focus:} This article belongs to the Special Issue AI and Blockchain as New Trigger in the Education Arena 
        \item \textbf{Methodology:} This research concept is based on a more careful perspective of the recognition and synthesis of information, which improve the validity of the review, accuracy and the generalizability of the results.
        \item \textbf{Key Findings:} 
        
         - Blockchain provides a digital and decentralised learning 
           infrastructure to all stakeholders 
           
         - Blockchain allows improving technology in terms of securing and 
           sharing authentic digital certificates
           
         - Blockchain enables the facilitation of knowledge organization
         
         - Blockchain improves the didactics reliant on data from shared 
           sources and innovation through educational projects
           
         - Blockchain applications in higher education rely on varying 
           environments of distinct actors, processes/structures and cannot 
           be analysed detached from it
           
         - Blockchain ensures privacy to the accreditation process
    \end{itemize}
    \vspace{0.2cm}
    \item \textbf{Paper 2:} Challenges of Using Blockchain in the Education Sector: A Literature Review 
    \begin{itemize}
        \item \textbf{Citation and Authors:} Abdulghafour Mohammad, Sergio Vargas, 2022 \cite{p2}
        \item \textbf{Research Focus:} Challenges of using Blockchain in education sector
        \item \textbf{Methodology:} Using combinations of keywords on ScienceDirect, Web of Science, Springer, IEEE Xplore, and MDPI to search for relevant data. Further relevant papers were also located by reviewing the references from previously identified papers.
        \item \textbf{Key Findings:} 
        
        - \textbf{Technological Barriers:} Immaturity, the complexity of integration, security issues, privacy, immutability and lack of flexibility, data unavailability

        - \textbf{Organizational Barriers:} Lack of adequate skills, financial barriers, lack of management commitment and support, legal issues and the lack of regulatory compliance, the market and ecosystem readiness, sustainability concerns
    \end{itemize}
    \vspace{0.2cm}
    \item \textbf{Paper 3:} Unlocking the power of blockchain in education: An overview of innovations and outcomes
    \begin{itemize}
        \item \textbf{Citation and Authors:} Amr El Koshiry, Entesar Eliwa, Tarek Abd El-Hafeez, Mahmoud Y. Shams, 2023 \cite{p3}
        \item \textbf{Research Focus:}
        
        - How blockchain technology has been defined in educational settings

        - How blockchain technology has been examined

        - Results of using blockchain technology in education
        \item \textbf{Methodology:} Collect data from SCOPUS and examine using bibliometric analysis
        \item \textbf{Key Findings:} 

        - Blockchain can change many industries, like finance, healthcare, and education

        - Adding Blockchain to education can improve efficiency, security, and credibility

        - Blockchain can make education more accessible and trustworthy for student and employers

        - Using Blockchain in education has challenges like adoption, technical knowledge, and regulation

        - Equipment is diverse and necessary for successful blockchain implementation in education

        - Blockchain can promote sustainability education and help individuals and organizations reach their sustainability goals

        - Organizations should add blockchain to their sustainability education programs to make them more efficient and transparent

        - Blockchain technology can transform education and make it more efficient and transparent
    \end{itemize}
    \vspace{0.2cm}
    \item \textbf{Paper 4:} Blockchain Adoption in higher-education institutions in India: Identifying the main challenges
    \begin{itemize}
        \item \textbf{Citation and Authors:} Sunita Dwivedi, Shinu Vig, 2023 \cite{p4}
        \item \textbf{Research Focus:} Explore the key challenges faced by educational institutes in adopting Blockchain technology
        \item \textbf{Methodology:} This research employed a qualitative methodology involving semi-structured interviews.
        \item \textbf{Key Findings:} 
        
        10 main challenges under three different dimensions:

        - Technological Dimension: Operational issues, security concerns, hardware related challenges, cost of new technology

        - Organisational Dimension: Attitudinal issues, human resource challenges, financial challenges

        - Environmental Dimension: Regulatory environment, stakeholders, competitive environment
    \end{itemize}
    \vspace{0.2cm}
    \item \textbf{Paper 5:} Barriers Affecting Higher Education Institutions' Adoption of Blockchain Technology: A Qualitative Study
    \begin{itemize}
        \item \textbf{Citation and Authors:} Abdulghafour Mohammad, Sergio Vargas, 2022 \cite{p5}
        \item \textbf{Research Focus:} Benefits, applications, and challenges of implementing Blockchain technology in higher education institutions
        \item \textbf{Methodology:} The qualitative research method with semi-structured interviews is used in this study.
        \item \textbf{Key Findings:} 

        - \textbf{Technological Barriers:} Immaturity, poor usability, lack of scalability, limited interoperability and standardization, integration complexity, security issues, privacy, immutability and lack of flexibility, data unavailability

        - \textbf{Organizational Challenges:} Lack of adequate skills, financial barriers, lack of management commitment and support

        - \textbf{Environmental Challenges:} Legal issues and lack of regulatory compliance, the market and ecosystem readiness, sustainability concern
    \end{itemize}
    \vspace{0.2cm}
    \item \textbf{Paper 6:} The Potential Blockchain Technology in Higher Education Learning Innovations in Era 4.0
    \begin{itemize}
        \item \textbf{Citation and Authors:} Primasatria Edastama, Suryari Purnama, Riya Widayanti, Lista Meria, Deva Rivelino, 2021 \cite{p6}
        \item \textbf{Research Focus:} Features and advantages of blockchain technology
        \item \textbf{Methodology:} Descriptive methods and literature study were used in this research.
        \item \textbf{Key Findings:} 

        - Basics of blockchain

        - Approval process in blockchain

        - Types of blockchain

        - Potential of blockchains in higher education
    \end{itemize}

    % Repeat for other papers...

\end{itemize}

\subsection{Critical Analysis of Each Paper}
\begin{itemize}
    \item \textbf{Paper 1:} Blockchain System in the Higher Education 
    \begin{itemize}
        \item \textbf{Strengths:}
        
        - Method of finding the research by analyze the reviewed studies and define thematic areas of research.

        - The findings of the research building for blockchain technology in the higher education sector.

        - The findings of the challenges and future research paths in blockchain
        \item \textbf{Weaknesses:} 

        - Finding uses existing knowledge on Blockchain application in the distinct domain of higher education

        - the review considered Blockchain as an umbrella keyword
        \item \textbf{Relevance: } It provide on how blockchain technology be implemented in the education system and provide explanation on challengers that need to be issued first.
    \end{itemize}
    \item \textbf{Paper 2:} Challenges of Using Blockchain in the Education Sector: A Literature Review
    \begin{itemize}
        \item \textbf{Strengths:} 
        
        - The findings of papers is by reviewing the references from previously identified papers and finding newer publications that referenced the cited article

        - Provide detailed challengers for blockchain technology
        \item \textbf{Weaknesses:} 

        - Choose to limit the number of irrelevant articles that were published many years ago, articles that were too general, and articles that did not focus on the research question

        - ruled out relevant articles written in languages other than English

        - Focused on the challenges of using blockchain technology in the education sector than solutions.
        
        \item \textbf{Relevance: } This paper provide on what are the many challengers that the use of blockchain in education system that need to be study
    \end{itemize}
    \item \textbf{Paper 3:} Unlocking the power of blockchain in education: An overview of innovations and outcomes
    \begin{itemize}
        \item \textbf{Strengths:} 

        - Providing how the blockchain would benefit in the education system

        - Detailed on the environment and equipment needed

        - The use of machine learning and blockchain in education system

        - Analyzed the initiatives of universities that have implemented blockchain technology in education
        \item \textbf{Weaknesses:} 

        - Focus on promoting sustainability in education and securing data and certification.

        - Did not examine the legal and ethical implications of implementing blockchain technology in education
        \item \textbf{Relevance: } The paper provide the benefit of using blockchain technology in the education system and provide example of usage of blockchain technology
    \end{itemize}
    \item \textbf{Paper 4:} Blockchain Adoption in higher-education institutions in India: Identifying the main challenges
    \begin{itemize}
        \item \textbf{Strengths:} 

        - Provide on the adoption of blockchain in HEIs categorized in the technological, organisational, and environmental.

        - Provide the challenges hindering blockchain adoption in higher education in India

        - Provide thoughts on institution should take in adopting blockchain technology
        \item \textbf{Weaknesses:} 

        - The analysis focused on the outlooks of higher administration teams and IT teams in private universities.

        - The research sample was mainly focus in India and did not consider from other countries.
        \item \textbf{Relevance: } The research provide the thoughts on the usage of blockchain technology on education from a country view point.
    \end{itemize}
    \item \textbf{Paper 5:} Barriers Affecting Higher Education Institutions' Adoption of Blockchain Technology: A Qualitative Study
    \begin{itemize}
        \item \textbf{Strengths:} 

        - Draws on results from a qualitative study method with semi-structured interviews from five countries.

        - Provide benefits and obstacles for the use of blockchain technology in the higher education sector
        
        \item \textbf{Weaknesses:} 

        - Small sample size

        - Employed only one research method for data collection.

        - The qualitative study results cannot be assumed to apply to all HEIs.
        \item \textbf{Relevance: } Provide the example of challenge of blockchain technology in education.
    \end{itemize}
    \item \textbf{Paper 6:} The Potential Blockchain Technology in Higher Education Learning Innovations in Era 4.0
    \begin{itemize}
        \item \textbf{Strengths:} 

        - Provide the potential of blockchain in education system
        \item \textbf{Weaknesses:} 

        - Did not address the many other problem with blockchain technology
        \item \textbf{Relevance: } The paper provide us on the potential use of blockchain technology in education section.
    \end{itemize}

\end{itemize}

\subsection{Comparative Analysis and Synthesis}
\begin{itemize}
    \item \textbf{Themes and Patterns:} The most common themes in these paper are blockchain technology are still an underdeveloped and face challenges from a technological, organisational and environmental stand point. The blockchain technology still have improvement that need to be make in order to be use in education sector.
    \item \textbf{Gaps in the Literature:} The gaps that in these paper is how legal and ethical the use of blockchain and how does this technology effect in education section.
    \item \textbf{Trends:} The emerging trends in the research area is the inclusion of machine learning in the use of blockchain technology in education sector. 
    \item \textbf{Synthesis:} The insights we got from the papers are in order for blockchain technology be accepted in education sector they must overcome the limitation currently facing for example the cost, the privacy, legal, and the usability.If they address this limitation the use of this technology can revolutionize the education system.
\end{itemize}

% 3. Discussion
\section{Discussion}
\begin{itemize}
    \item \textbf{Evaluation of the Literature:} The reviewed papers highlight both the potential and challenges of using blockchain in higher education. Blockchain can improve security, transparency, and digital record management. But it also has it's weaknesses including technological barriers, lack of skills, and regulatory and under-developed issues. The research question is addressed by examining both the potential of blockchain and the obstacles to its implementation. However, the focus in all the research papers tend to be more on theoretical applications and less on practical utilisation.
    \item \textbf{Gaps in Research:} lack of real-world case studies showing how blockchain is being used in universities. Most studies are theoretical or based on small-scale interviews.  Additionally, concerns about student data privacy are not fully explored.
    \item \textbf{Implications for Future Research:} Studies should focus on how blockchain is used in real educational settings, including more case studies from various institutions. There should also be more research on how blockchain affects data privacy. Studies should also focus on how blockchain can integrate with existing educational technologies, such as Learning Management Systems (LMS).
\end{itemize}

% 4. Conclusion
\section{Conclusion}
In conclusion, our research methodology employed in this study was designed for a challenge to make blockchain system usable in higher education. The research on blockchain technology in higher education highlights its potential to enhance data security, transparency, and sustainability, but significant challenges persist. Key barriers include technological issues like immaturity and poor usability, organizational hurdles such as inadequate skills and financial constraints, and environmental challenges related to legal and regulatory compliance. While the benefits of blockchain, such as secure data management and innovative learning solutions, are promising, the focus has predominantly been on technological challenges, leaving gaps in understanding the organizational and environmental aspects. Future research should prioritize addressing these gaps, exploring cultural and regional impacts, and developing holistic, sustainable ecosystems that involve all stakeholders.

In summary, the methodological rigor applied into this research ensures that the findings are reliable and relevant, offering guidance for future studies aimed at blockchain system usage in every kind of purposes.

Future research should prioritize addressing these gaps, exploring cultural and regional impacts, and developing holistic, sustainable ecosystems that involve all stakeholders.
% References
\newpage
\bibliographystyle{apacite} % Choose the appropriate bibliography style (e.g., APA, IEEE)

\bibliography{MyBib} % Make sure to include a MyBib.bib file with your references

% Appendices (if necessary)
\appendix
\section{Appendix A: Member Contributions}

\begin{tabular}{|c|c|c|}
    \hline
    Name & Contribution & Percentage \\
    \hline
    Rusyaidi Azri& \parbox{7cm}{-Introduction\\-Critical Analysis of Each Paper\\-Comparative Analysis and Synthesis} & 30 \\
    \hline
    Toh Jing Wei&\parbox{7cm}{-Abstract\\-Overview of Selected Papers} & 30 \\
    \hline
    Muhammad Qayyim Hannan& -Discussion & 20 \\
    \hline
    Lai Zi Xuan& -Conclusion & 20 \\
    \hline
     
\end{tabular}


\end{document}